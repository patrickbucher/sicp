\documentclass[a4paper]{scrartcl}
\usepackage{amsmath}

\begin{document}
\setlength\parindent{0pt}

\section*{SICP: Ex. 2.9, p. 95}

\subsection*{Addition}

Given:

\begin{align}
a &= 1, w_a = 0.1, a = [0.9;1.1] \\
b &= 2, w_b = 0.2, b = [1.8;2.2] \\
c_l &= 0.9 + 1.8 = 2.7 \\
c_u &= 1.1 + 2.2 = 3.3 \\
w_c &= \frac{c_u - c_l}{2} = \frac{3.3 - 2.7}{2} = \frac{0.6}{2} = 0.3 \\
w_c &= w_a + w_b = 0.1 + 0.2 = 0.3
\end{align}

Same width, switched values:

\begin{align}
p &= 2, w_p = 0.1, p = [1.9;2.1] \\
q &= 1, w_q = 0.2, q = [0.8;1.2] \\
r_l &= 1.9 + 0.8 = 2.7 \\
r_u &= 2.1 + 1.2 = 3.3 \\
w_r &= \frac{r_u - r_l}{2} = \frac{3.3 - 2.7}{2} = \frac{0.6}{2} = 0.3 \\
w_r &= w_p + w_q = 0.1 + 0.2 = 0.3
\end{align}

For addition, only the width matters, not the actual numbers.

\subsection*{Multiplication}

For multiplication, all four combinations have to be tried:

\begin{align}
a &= 1, w_a = 0.1, a = [0.9;1.1] \\
b &= 2, w_b = 0.2, b = [1.8;2.2] \\
c_1 &= a_l * b_l = 0.9 \times 1.8 = 1.62 \\
c_2 &= a_l * b_u = 0.9 \times 2.2 = 1.98 \\
c_3 &= a_u * b_l = 1.1 \times 1.8 = 1.98 \\
c_4 &= a_u * b_u = 1.1 \times 2.2 = 2.42 \\
c_l &= \text{min}(c_1, c_2, c_3, c_4) = 1.62 \\
c_u &= \text{max}(c_1, c_2, c_3, c_4) = 2.42 \\
w_c &= \frac{c_u - c_l}{2} = \frac{2.42 - 1.62}{2} = \frac{0.8}{2} = 0.4
\end{align}

Same width, switched values:

\begin{align}
p &= 2, w_p = 0.1, p = [1.9;2.1] \\
q &= 1, w_q = 0.2, q = [0.8;1.2] \\
r_1 &= p_l * q_l = 1.9 \times 0.8 = 1.52 \\
r_2 &= p_l * q_u = 1.9 \times 1.2 = 2.28 \\
r_3 &= p_u * q_l = 2.1 \times 0.8 = 1.68 \\
r_4 &= p_u * q_u = 2.1 \times 1.2 = 2.52 \\
r_l &= \text{min}(r_1, r_2, r_3, r_4) = 1.52 \\
r_u &= \text{max}(r_1, r_2, r_3, r_4) = 2.52 \\
w_r &= \frac{r_u - r_l}{2} = \frac{2.52 - 1.52}{2} = \frac{1.0}{2} = 0.5
\end{align}

The product's width cannot be determined by the width of the two factors alone,
because it also depends on the factors itself!

\end{document}
