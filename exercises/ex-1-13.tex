\documentclass[a4paper]{scrartcl}
\usepackage{amsmath}

\begin{document}
\setlength\parindent{0pt}

\section*{SICP: Ex. 1.13, p. 42}

\subsection*{Exercise}

Given:

$$
\phi = \frac{1 + \sqrt{5}}{2},
\psi = \frac{1 - \sqrt{5}}{2},
$$

And:

$$
Fib(n) = 
\left\{
    \begin{array}{lll}
        0 & n = 0 \\
        1 & n = 1 \\
        \text{Fib}(n-1) + \text{Fib}(n-2) & n > 1
    \end{array}
\right.
$$

Proof that:

$$ \text{Fib}(n) = \frac{\phi^n-\psi^n}{\sqrt{5}} $$

\subsection*{Proof}

\begin{align}
    \text{Fib}(2) & = \text{Fib}(1) + \text{Fib}(0) \\
    \frac{\phi^2-\psi^2}{\sqrt{5}} & = \frac{\phi-\psi}{\sqrt{5}} + \frac{\phi^0-\psi^0}{\sqrt{5}} \\
    \frac{\phi^2-\psi^2}{\sqrt{5}} & = \frac{\phi-\psi}{\sqrt{5}} \\
    \phi^2 - \psi^2 & = \phi - \psi \\
    (\phi - \psi)(\phi + \psi) & = \phi - \psi
\end{align}

With the definitions of $\phi$ and $\psi$:

\begin{align}
    \left(\frac{1+\sqrt{5}}{2}\right)^2 - \left(\frac{1-\sqrt{5}}{2}\right)^2 & =  \frac{1+\sqrt{5}}{2} - \frac{1-\sqrt{5}}{2} \\
    \frac{(1+\sqrt{5})^2 - (1-\sqrt{5})^2}{4} & = \frac{1+\sqrt{5}-1+\sqrt{5}}{2} \\
    \frac{1+2\sqrt{5}+5-(1-2\sqrt{5}+5)}{4} & = \frac{2\sqrt{5}}{2} \\
    \frac{6 + 2\sqrt{5} - 6 + 2\sqrt{5}}{4} & = \sqrt{5} \\
    \frac{4\sqrt{5}}{4} & = \sqrt{5} \\
    \sqrt{5} & = \sqrt{5} \qquad \text{q.e.d.}
\end{align}

\end{document}
